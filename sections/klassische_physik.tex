\documentclass[../document.tex]{subfiles}
\begin{document}

\subsection{Die newtonschen Gesetze}


\textbf{Das erste newtonsche Gesetz} oder auch Trägheitsprinzip

\begin{quote}
„Ein Körper verharrt im Zustand der Ruhe oder der gleichförmig geradlinigen Translation, sofern er nicht durch einwirkende Kräfte zur Änderung seines Zustands gezwungen wird.“
\end{quote}

Die Geschwindigkeit $\vec{v}$ ist also in Betrag und Richtung konstant, sofern die Summe der einwirkenden Kräfte null ist. Eine Änderung kann nur durch Ausübung einer Kraft von außen erreicht werden, beispielsweise durch die Gravitationskraft.


\textbf{Das zweite newtonsche Gesetz} oder auch Aktionsprinzip

\begin{quote}
„Die Änderung der Bewegung ist der Einwirkung der bewegenden Kraft proportional und geschieht nach der Richtung derjenigen geraden Linie, nach welcher jene Kraft wirkt.“
\end{quote}

Die beschreibende Gleichung lautet $\vec{F} = m\vec{a} = \dot{\vec{p}}$ und heißt häufig \textbf{Grundgleichung der Mechanik} 


\textbf{Das dritte newtonsche Gesetz} oder auch Wechselwirkungsprinzip

\begin{quote}
„Kräfte treten immer paarweise auf. Übt ein Körper A auf einen anderen Körper B eine Kraft aus (actio), so wirkt eine gleich große, aber entgegen gerichtete Kraft von Körper B auf Körper A (reactio).“
\end{quote}

Die beschreibende Gleichung lautet $\vec{F_1} = -\vec{F_2}$ 

Das Wechselwirkungsprinzip lässt sich auch so ausdrücken, dass in einem abgeschlossenen System die Summe aller Kräfte gleich Null ist.

\textbf{Das Superpositionsprinzip der Kräfte}

Das Superpositionprinzip der Mechanik wurde als Zusatz zu den Bewegungsgesetzen beschrieben und lautet wie folgt:

\begin{quote}
„Wirken auf einen Punkt (oder einen starren Körper) mehrere Kräfte ${\vec  {F_{1}}},{\vec  {F_{2}}},\dots,{\vec  {F_{n}}}$, so addieren sich diese vektoriell zu einer resultierenden Kraft ${\vec {F}}$ auf.“
\end{quote}

Als Gleichung $\vec{F_{{res}}}=\vec  {F_{1}}+\vec  {F_{2}}+\dots+\vec  {F_{n}}$


\subsection{Arbeit, Energie, Leistung}
\subsubsection{Energie}

Unter Energie versteht man die Fähigkeit eines Körpers, Arbeit zu verrichten. Energie kann also auch als "gespeicherte Arbeit" gesehen werden. (nicht allgemein gültig). Das Formelzeichen von Energie ist $E$. Die SI-Einheit von Energie ist Joule $J = N * m$. Es gilt der Energieerhaltungssatz: "Die Gesamtenergie in einem abgeschlossenen System kann weder vermehrt noch vermindert werden."

Verschiedene Arten von Energie:
\begin{itemize}
 \item \textbf{Kinetische Energie} immer proportional zur Masse und quadratisch zur Geschwindigkeit
 \begin{itemize}
  \item Bewegungsenergie $E_{\mathrm {kin} }\,=\,{\frac {1}{2}}mv^{2}\,$
  \item Rotationenergie  $E_{\mathrm {rot} }={\frac {1}{2}}\cdot J_{x}\cdot \omega ^{2}$
 \end{itemize}
 \item \textbf{potentielle Energie} $E_{{\mathrm  {pot}}}=F_{{\mathrm  G}}\,h=m\,g\,h$ (Nur gültig für ungefähre Berechnung in Erdnähe \footnote{Siehe allgemeinere Beschreibung \url{https://de.wikipedia.org/wiki/Potentielle\_Energie\#Allgemeinere\_Beschreibung}})
 \item \textbf{thermische Energie} ist in der Bewegung der Atome oder Moleküle eines Stoffes gespeichert. 
\end{itemize}

$J_x$: Trägheitsmoment des Körpers um die x-Achse\newline
$\omega$: Winkelgeschwindigkeit.\newline
$m$: Masse\newline
$g$: Erdbeschleunigung\newline
$h$: Höhe über dem Boden

\subsubsection{Arbeit}
Grob gesagt entspricht Arbeit der Übertragung von Energie. Das Formelzeichen von Arbeit ist $W$, die SI-Einheit für Arbeit ist identisch mit der von Energie: das Joule $1 J = 1 Nm = 1 Ws$. 
\sphericalangle a
Definiert ist Arbeit über folgende Formel:
$W={\vec  F}\cdot {\vec  s}=|{\vec  F}|\,|{\vec  s}|\,\cos \sphericalangle \left({\vec  F},{\vec  s}\right)\,$


\subsection{Optik}



\end{document}
