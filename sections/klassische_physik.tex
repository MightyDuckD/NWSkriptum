\documentclass[../document.tex]{subfiles}
\begin{document}

\subsection{Die newtonschen Gesetze}

\subsubsection{Das erste newtonsche Gesetz oder auch Trägheitsprinzip}

\begin{quote}
„Ein Körper verharrt im Zustand der Ruhe oder der gleichförmig geradlinigen Translation, sofern er nicht durch einwirkende Kräfte zur Änderung seines Zustands gezwungen wird.“
\end{quote}

Die Geschwindigkeit $\vec{v}$ ist also in Betrag und Richtung konstant, sofern die Summe der einwirkenden Kräfte null ist. Eine Änderung kann nur durch Ausübung einer Kraft von außen erreicht werden, beispielsweise durch die Gravitationskraft.


\subsubsection {Das zweite newtonsche Gesetz oder auch Aktionsprinzip}

\begin{quote}
„Die Änderung der Bewegung ist der Einwirkung der bewegenden Kraft proportional und geschieht nach der Richtung derjenigen geraden Linie, nach welcher jene Kraft wirkt.“
\end{quote}

Die beschreibende Gleichung lautet $\vec{F} = m\vec{a} = \dot{\vec{p}}$ und heißt häufig \textbf{Grundgleichung der Mechanik} 


\subsubsection {Das dritte newtonsche Gesetz oder auch Wechselwirkungsprinzip}

\begin{quote}
„Kräfte treten immer paarweise auf. Übt ein Körper A auf einen anderen Körper B eine Kraft aus (actio), so wirkt eine gleich große, aber entgegen gerichtete Kraft von Körper B auf Körper A (reactio).“
\end{quote}

Die beschreibende Gleichung lautet $\vec{F_1} = -\vec{F_2}$ 

Das Wechselwirkungsprinzip lässt sich auch so ausdrücken, dass in einem abgeschlossenen System die Summe aller Kräfte gleich Null ist.

\subsubsection {Das Superpositionsprinzip der Kräfte}

Das Superpositionprinzip der Mechanik wurde als Zusatz zu den Bewegungsgesetzen beschrieben und lautet wie folgt:

\begin{quote}
„Wirken auf einen Punkt (oder einen starren Körper) mehrere Kräfte ${\vec  {F_{1}}},{\vec  {F_{2}}},\dots,{\vec  {F_{n}}}$, so addieren sich diese vektoriell zu einer resultierenden Kraft ${\vec {F}}$ auf.“
\end{quote}

Als Gleichung $\vec{F_{{res}}}=\vec  {F_{1}}+\vec  {F_{2}}+\dots+\vec  {F_{n}}$


\subsection{Arbeit, Energie, Leistung}
\subsubsection{Energie}

Unter Energie versteht man die Fähigkeit eines Körpers, Arbeit zu verrichten. Energie kann also auch als "gespeicherte Arbeit" gesehen werden. (nicht allgemein gültig). Das Formelzeichen von Energie ist $E$. Die SI-Einheit von Energie ist Joule $J = N * m$. Es gilt der Energieerhaltungssatz: "Die Gesamtenergie in einem abgeschlossenen System kann weder vermehrt noch vermindert werden."

Verschiedene Arten von Energie:
\begin{itemize}
 \item \textbf{Kinetische Energie} immer proportional zur Masse und quadratisch zur Geschwindigkeit
 \begin{itemize}
  \item Bewegungsenergie $E_{\mathrm {kin} }\,=\,{\frac {1}{2}}mv^{2}\,$
  \item Rotationenergie  $E_{\mathrm {rot} }={\frac {1}{2}}\cdot J_{x}\cdot \omega ^{2}$
 \end{itemize}
 \item \textbf{potentielle Energie} $E_{{\mathrm  {pot}}}=F_{{\mathrm  G}}\,h=m\,g\,h$ (Nur gültig für ungefähre Berechnung in Erdnähe \footnote{Siehe allgemeinere Beschreibung \url{https://de.wikipedia.org/wiki/Potentielle\_Energie\#Allgemeinere\_Beschreibung}})
 \item \textbf{thermische Energie} ist in der Bewegung der Atome oder Moleküle eines Stoffes gespeichert. 
\end{itemize}

$J_x$: Trägheitsmoment des Körpers um die x-Achse\newline
$\omega$: Winkelgeschwindigkeit.\newline
$m$: Masse\newline
$g$: Erdbeschleunigung\newline
$h$: Höhe über dem Boden

\subsubsection{Arbeit}
Grob gesagt entspricht Arbeit der Übertragung von Energie. Das Formelzeichen von Arbeit ist $W$, die SI-Einheit für Arbeit ist identisch mit der von Energie: das Joule $1 J = 1 Nm = 1 Ws$. 

Arbeit ist das Skalarprodukt aus Weg und Kraft: Wenn auf einen Körper entlang einer geraden Strecke von A nach B eine konstante Kraft $\vec F$ wirkt, dann wird auf dem Körper die Arbeit
$W={\vec  F}\cdot {\vec  s}=|{\vec  F}|\,|{\vec  s}|\,\cos \sphericalangle \left({\vec  F},{\vec  s}\right)\,$ verrichtet.

Einige Beispiele für Arbeit sind
\begin{itemize}
	\item \textbf{Hubarbeit} ist die Arbeit die verrichtet werden muss um einen Körper der Masse $m$ in einem homogenen Schwerefeld mit der Beschleunigung $a$ um die Hubhöhe $h$ zu heben: Die benötigte Kraft entspricht $F = mg$ und die damit geleistete Hubarbeit $W_h = Fs = \,m\,g\,h$ 
	\item \textbf{Spannarbeit} ist die Arbeit die beim spannen einer Feder auftritt: $W_s = \frac {1}{2} D s^2$ , wobei $D$ die Federkonstante ist.
	\item \textbf{Beschleunigungsarbeit} ist die Arbeit die bei der Beschleunigung der Masse $m$ aus der Ruhe auf eine Geschwindigkeit von $v$ benötigt wird: $W_b = \frac{1}{2} m v^2$
\end{itemize}
\subsection{Optik}

Die Optik wird auch Lehre vom Licht genannt und beschäftigt sich mit der Ausbreitung von Licht sowie dessen Wechselwirkungen. 

\subsubsection {Licht, Schatten und Reflexion}
Wichtige Begriffe
\begin{itemize} 
	\item \textbf{Kernschatten} ist der dunkelste Teil des Schatten.
	\item \textbf{Halbschatten} der Teil welcher nur von einem Teil der realen Lichtquelle beleuchtet wird.
	\item \textbf{Einfallswinkel $\alpha$ = Reflexionswinkel $\beta$} tritt licht unter winkel $\alpha$ zum Lot auf eine glatte Grenzfläche auf, wird ein Teil unter dem Winkel $\beta$ reflektiert
	\item \textbf{diffuse und reguläre Reflexion}, im Gegensatz zur regulären Reflexion (z.B: bei Spiegel) wird bei der diffusen Reflexion das licht in alle Richtungen zerstreut und ist somit von jedem Punkt im Raum erkennbar (z.B Wand).
	\item \textbf{Retroreflexion} tritt ein wenn $\alpha = \beta = 0$ ist.
\end{itemize}

\subsubsection{Brechung}

Der Brechungsindex $n = \frac{c_0}{c_M}$ eines Medium gibt an, um welchen Faktor die Wellenlänge und die Phasengeschwindigkeit des Lichts kleiner sind als im Vakuum. Zum Beispiel bei Luft $n_{Luft} = \frac{c_0}{c_{Luft}} \approx \frac{299792458}{299703000} \approx 1,0003$

Das \textbf{Brechungsgesetz von Snellius} beschreibt die Richtungsänderung einer Welle bei dem Übergang in ein anderes Medium. 
$\frac{\sin\alpha}{\sin\beta} = \frac{c_1}{c_2} = \frac{\lambda_1}{\lambda_2} = \frac{n_2}{n_1}$

\begin{itemize}
	\item Die \textbf{Brechung zum Lot} tritt auf wenn Licht von einem dünneren Medium $n_1$ in ein dichteres Medium $n_2$ eintritt. Es gilt $n_2 > n_1$ und $\beta < \alpha$
	\item Die \textbf{Brechung vom Lot} tritt auf wenn Licht von einem dichteren Medium $n_1$ in ein dünneres Medium $n_2$ eintritt. Es gilt $n_2 < n_1$ und $\beta > \alpha$
\end{itemize}

Merkhilfe: Wenn man sich das Brechungsgesetz von Snellius anschaut, dann sind sowohl $n_1$ als auch $\sin \beta$ im Nenner, wenn also der eine Wert größer als sein Zähler ist muss auch der andere Wert größer als sein Zähler sein um die Gleichung zu erfüllen.

Eine \textbf{Totalreflexion} tritt auf, wenn bei der Brechung vom Lot der Brechungswinkel $\beta$ größer als $90^\degree$ beträgt. Dabei tritt der Lichtstrahl nicht mehr in das optisch dünnere Medium ein, sondern verbleibt im optisch dichteren Medium (Wobei nun Austrittswinkel gleich Eintrittswinkel gilt?). Bei einer Totalreflexion findet keine Brechung mehr statt.

Der \textbf{Grenzwinkel der Totalreflexion } $\alpha_{Grenz} = arcsin(\frac{n_2}{n_1}) = arcsin(\frac{c_1}{c_2}) = arcsin(\frac{\lambda_1}{\lambda_2})$ ist jener Winkel, bei dem der Brechungswinkel $\beta = 90^\degree$ ist \footnote{Formel ergibt sich weil bei $\sin\beta^\degree$ und $\beta = 90^\degree$ der $\beta$ Teil vom Gesetzt wegfällt}. 

Dadurch das 
\end{document}
